\chapter{Design}

\section{Introduction}
This chapter aims to provide a detailed overview of the software architecture and database design of the project. It is essential for understanding the organization and structure of the system, as well as the design choices made to ensure the efficiency, scalability, and robustness of the software.

The design of the software architecture focuses on the organization and distribution of software components, defining roles, responsibilities, and interactions among them. Key architectural decisions guiding the project's development will be presented within this context.

Additionally, the database design will be examined, with particular attention to the decision to use a NoSQL database like MongoDB. This decision was motivated by the need to adapt to the specific requirements of the project, including flexible management of unstructured data and horizontal scalability.


\section{Software Architecture}

\begin{figure}[ht!]
    \centering
    \includegraphics[width=0.8\textwidth]{images/2_analisys/FL_proj_Arch_FINAL.png}
    \caption{System Architecture}
    \label{fig:system_architecture}
\end{figure}


\newpage
\section{Database Design}

\subsection{MongoDB}
\subsubsection{Collections}
\textbf{ExpConfig document example:} \begin{verbatim}
        {
            "_id": {
              "$oid": "6613f8b7aed2e52b006dea10"
            },
            "name": "TestConfig",
            "algorithm": "fcmeans",
            "codeLanguage": "python",
            "clientSelectionStrategy": "probability",
            "clientSelectionRatio": 1,
            "minNumberClients": 2,
            "stopCondition": "max_number_rounds",
            "stopConditionThreshold": 5,
            "maxNumberOfRounds": 10,
            "parameters": {
              "targetFeature": "16",
              "lambdaFactor": "2",
              "numFeatures": "16",
              "seed": "10",
              "numClusters": "10"
            },
            "creationDate": {
              "$date": "2024-04-08T14:01:27.232Z"
            },
          }
    \end{verbatim}
\textbf{Experiment document example:} \begin{verbatim}
        {
            "_id": {
              "$oid": "661c3d780bb4be3bd9b891b9"
            },
            "name": "ExpTest",
            "expConfig": {
              "_id": {
                "$oid": "6613f8b7aed2e52b006dea10"
              },
              "name": "TestConfig",
              "algorithm": "fcmeans"
            },
            "creationDate": {
              "$date": "2024-04-14T20:32:56.022Z"
            },
            "status": "FINISHED",
            "flExpId": "\"d9d1bc7c-d733-4219-b4fb-16a3849db323\"",
            "modelPath": "\\FL_models\\exp_661c3d780bb4be3bd9b891b9.bin"
          }

    \end{verbatim}

\newpage
\textbf{ExperimentMetrics document example:} \begin{verbatim}
        {
            "_id": {
              "$oid": "66144b5337a2fd7f67582f67"
            },
            "expId": "661c3d780bb4be3bd9b891b9",
            "type": "STRATEGY_SERVER_METRICS",
            "hostMetrics": {
              "cpuUsagePercentage": 5,
              "memoryUsagePercentage": 9.27
            },
            "modelMetrics": {
              "FRO": 845.7339394664009
            },
            "timestamp": {
              "$date": "1970-01-20T19:43:26.034Z"
            },
            "round": 1,
          }
    \end{verbatim}

\textbf{User document example:} \begin{verbatim}
    {
        "_id": {
          "$oid": "6611252030f96a50aebda458"
        },
        "email": "admin@example.com",
        "password": "P@ssw0rd",
        "description": "example description of a user",
        "creationDate": {
          "$date": "2024-04-06T10:34:08.669Z"
        },
        "role": "admin",
        "configurations": [
          "6613f8b7aed2e52b006dea10"
        ],
        "experiments": [{
            "_id": {
              "$oid": "661c3e800bb4be3bd9b891da"
            },
            "name": "ExpTest",
            "config": "TestConfig",
            "creationDate": {
              "$date": "2024-04-14T20:32:56.022Z"
            }
          }]
      }

    \end{verbatim}
\newpage
\section{Message Handler}

\subsection{JInterface for Message Passing}
The message handler is implemented using the Erlang programming language. Erlang is a functional
programming language designed for building scalable and fault-tolerant systems. It is
particularly well-suited for building distributed systems, thanks to its lightweight
processes and built-in support for message passing. In this project, it's utilized the 
JInterface library, which allows to write Java code that can communicate with Erlang 
processes to send and receive messages. Rather than using directly Erlang nodes with Erlang language and communicate with them
through web socket, the choice of adopting JInterface library provides manipulation of messages directly in 
Java, storing the information contained in the messages in database and sending them to frontend.

\subsection{Message Structure}
Almost all the messages sent from the FLang Infrastructure to the Erlang message handler are tuples containing one atom to specify the
message type and a string for the message body in Json format. The json string contains the information about the message, such as the timestamp,
the round number, the client id, and the metrics. The following are some examples of message bodies in json format:
\begin{itemize}
    \item Experiment Queued:
    \begin{verbatim}
        {
            "type": "experiment_queued",
            "timestamp": "1712206255"
        }
    \end{verbatim}

    \item Worker Ready:
    \begin{verbatim}
        {
            "type": "worker_ready",
            "timestamp": "1712206257",
            "client_id": "1"
        }
    \end{verbatim}

    \item Strategy Server Ready:
    \begin{verbatim}
        {
            "type": "strategy_server_ready",
            "timestamp": "1712206257"
        }
    \end{verbatim}

    \item All Workers Ready:
    \begin{verbatim}
        {
            "type": "all_workers_ready",
            "timestamp": "1712206257"
        }
    \end{verbatim}

    \item Start Round:
    \begin{verbatim}
      {
          "type": "start_round",
          "timestamp": "1712206257",
          "round": "1"
      }
    \end{verbatim}

    \newpage

    \item Woker Metrics:
    \begin{verbatim}
      {
        "type": "woker_metrics",
        "timestamp": "1712206258",
        "round": "1"
        "client_id": "1"
        "hostMetrics": {
          "cpuUsagePercentage":39.4,
          "memoryUsagePercentage": 52.69
        }
        "modelMetrics":{
          "ARI":"0.10764575464353877"
        }
      }
    \end{verbatim}

    \item End Round:
    \begin{verbatim}
      {
        "type": "end_round",
        "timestamp": "1712206260",
        "round": "1"
      }
    \end{verbatim}

    \item Strategy Server Metrics:
    \begin{verbatim}
      {
        "type": "strategy_server_metrics",
        "timestamp": "1712206264",
        "round": "2",
        "hostMetrics": {
          "cpuUsagePercentage":39.4,
          "memoryUsagePercentage": 52.69
        }
         "modelMetrics":{
          "FRO":"0.13237183370436725"
        }
      }
    \end{verbatim}
   
\end{itemize}

\subsection{}
The trained model is sent at the end of the experiment in a special message that consists of a tuple with arity 3, \
where the first element is the message type's atom ``fl\_end\_str\_run'',
the second element is the identifier used by the FLang Infrastructure to identify the experiment, and the third element is the model in binary format.
\\
The message structure is designed to be flexible and extensible, allowing for easy integration of new message types and
additional information as needed. This design choice enables the system to adapt to changing requirements and accommodate
future enhancements without significant modifications to the existing codebase.

\subsection{Description of the Erlang Message Handler Module}
The Erlang message handler module is a crucial component of the system responsible for managing incoming messages, processing them accordingly, and facilitating communication between different parts of the distributed system. It encapsulates the logic for handling various types of messages, such as error notifications, stop signals, and data updates, ensuring proper routing and processing. Additionally, the module provides interfaces for sending and receiving messages, abstracting the underlying communication mechanisms and enabling seamless integration with other system components. Its robust design and fault-tolerant features contribute to the overall reliability and performance of the distributed system.