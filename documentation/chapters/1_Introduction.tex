\chapter{Introduction and Project Overview}

\section{Context and Project Objective}

The goal of this project is to develop a system to manage Federated Learning (FL) experiments. 
FL is a decentralized machine learning approach where multiple devices collaborate to train a 
shared model while keeping their data locally. The project aims to provide a graphic interface
 with a web console to run FL experiments, enabling users to monitor their progress and analyze 
 results. The system will use the Federated Learning Director (FL Director) to coordinate the 
 execution of experiments among the devices. The FL Director is an Erlang node that handle the 
 request of new experiments to propagate to the participants' devices. The system will be designed to support
  concurrent execution of multiple FL experiments, real-time analytics, and flexible storage
   of experiment statistics. The project will adopt the Model-View-Controller (MVC) pattern 
   to structure the Web Console, promoting separation of concerns and maintainability. The 
   system will utilize DocumentDB for flexible storage of experiment statistics and its
    horizontal scalability. Concurrent execution of experiments will be implemented using Java 
    threads and ExecutorService to optimize resource utilization. WebSocket communication 
    will be established for real-time data exchange, enabling seamless interaction between
     the frontend and backend. The project will define message formats and outline the structure
      of Erlang nodes for efficient communication. The system will provide a user-friendly
       Web Console to initiate and manage FL experiments, as well as centralized access to 
       experiment statistics for easy monitoring and analysis.

\section{Project Key Points}
\begin{itemize}
    \item Adopt the MVC (Model-View-Controller) pattern to structure the Web Console, promoting separation of concerns and maintainability.
    \item Utilize DocumentDB for flexible storage of experiment statistics and its horizontal scalability.
    \item Implement concurrent execution of experiments using Java threads and ExecutorService to optimize resource utilization.
    \item Establish WebSocket communication for real-time data exchange, enabling seamless interaction between the frontend and backend.
    \item Define message formats and outline the structure of Erlang nodes for efficient communication, ensuring reliability and scalability.
    \item System with a graphical interface and web console for executing FL experiments.
    \item Coordination of experiments through FL Director, an Erlang node.
    \item Real-time analytics and flexible storage of experiment statistics.
    \item Centralized access to experiment statistics for monitoring and analysis.
\end{itemize}
