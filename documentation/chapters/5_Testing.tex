\chapter{Testing}

Testing methodologies are used to ensure about the reliability, correctness, functionality and quality of the Federated Learning Web Console.
 In this chapter, the testing methodologies used in the project are described. The testing methodologies are divided into two main categories: structural testing and functional testing.

\section{Structural Testing}

%The structural testing performed on the project is described in this section.
Structural testing, also known as white-box testing is applied to the project to ensure that the implemented code is working as expected and evaluate the internal structure of the system.
For primary structure testing JUnit testing is applied as a testing methodology.

\subsection{JUnit Testing}

The JUnit testing performed on the project of FL Web Console. The JUnit testing is applied to various classes like DAOs and Services to check whether the implemented code is working as expected or not and specified
requirement are hold by the methods. Some examples of the JUnit testing that performed on the classes:

\subsubsection{UserDAO}
The UserDAO class is an important component of the project which is responsible for interacting with the database to handle data related with users. With the help of the JUnit tests
different scenarios are tested to ensure that the implemented code is working as expected and the requirements are fulfilled. This scenarios are including creating new user, deleting existing user, finding user by some criterias.
These tests show the correctness of the CRUD operations of the UserDAO class. Below it can be seen an example of performing JUnit test for some methods in the UserDAO.

\begin{figure}[ht!]
    \centering
    \includegraphics[width=0.8\textwidth]{images/5_testing/userdao-test}
    \caption{Testing the method of findListOfConfigurationsByEmail() in UserDAO class}
    \label{fig:u_dao_test}
\end{figure}

\begin{figure}[ht!]
    \centering
    \includegraphics[width=0.8\textwidth]{images/5_testing/userdao-test-result}
    \caption{Testing result for the method of findListOfConfigurationsByEmail() in UserDAO class}
    \label{fig:u_dao_test_result}
\end{figure}

With this JUnit test method, the findListOfConfigurationsByEmail() method of the UserDAO class is tested. The test is performed by finding all the related configuration that are belong to the user with that email.
The test is successful and the expected result is returned as a list of configurations.

\newpage
\subsubsection{ExperimentDAO}
Experiment DAO is another important class of the project which is responsible for interacting with the database to handle data related with experiments. JUnit tests are created to ensure that the experiment related functions are
working as expected and the requirements are fulfilled. The tests are including creating new experiment, updating existing experiment, deleting existing experiment, finding experiment by some criterias. Below it can be seen an example test method for update an experiment.

\begin{figure}[ht!]
    \centering
    \includegraphics[width=0.8\textwidth]{images/5_testing/experimentdao-test}
    \caption{Testing the method of update() in ExperimentDAO class}
    \label{fig:e_dao_test}
\end{figure}

\newpage
\subsubsection{ConfigurationDAO}

Test class for Configuration DAO is another example of JUnit test that is performed on the project. The Configuration DAO class is responsible for interacting with the database to handle storage and retrival of the system experiment configuration.
Implementing JUnit for this class guarantees that system operates the data in an intended way for configuration class. Below there is an example test and test result for saveAndRetrieve() method of Configuration DAO class. The test is performed by
saving a configuration and then retrieving it from the database. The test is successful and the expected result is returned.

\begin{figure}[ht!]
    \centering
    \includegraphics[width=0.8\textwidth]{images/5_testing/expconfigdao-test}
    \caption{Testing the method of saveAndRetrieve() in Configuration DAO class}
    \label{fig:c_dao_test}
\end{figure}

\begin{figure}[ht!]
    \centering
    \includegraphics[width=0.8\textwidth]{images/5_testing/expconfigdao-test-result}
    \caption{Testing result for the method of saveAndRetrieve() in Configuration DAO class}
    \label{fig:c_dao_test_result}
\end{figure}


\newpage
\section{Functional Testing}

%The functional testing performed on the project is described in this section.
Functional testing, also known as black-box testing is applied to the project to evaluate the system behaviour that needs to fulfill functional requirements.
Functional testing helps to ensure that user expectations are provided in a right way. The functional testing is performed by creating test cases for the system.\\

Test cases are identified according to the functional requirements. It shows how the system should behave in different scenarios that are both normal and anormal.
Those test cases are including user authentication, creating new configuration, creating new experiment, deleting experiment, finding experiment, finding configuration, deleting configuration, etc.
Below there is a table that shows some examples of the test cases that are created for the project. As a result of the test cases, it can be said that the Federated Learning Web Console provides all the
necessary functionalities and meets user expectations.

\subsection{Test Cases}

%\begin{table}[ht!]
%    \centering
%    \caption{Test case}
%    \begin{tabularx}{\textwidth}{|Sl|S{X}|S{X}|S{X}|S{X}|S{X}|}
%        \hline
%        \textbf{Id} & \textbf{Description} & \textbf{Input} & \textbf{E. Output} & \textbf{Output} & \textbf{Outcome} \\ \hline
%        U\_T\_01    & Lorem Ipsum          & Lorem Ipsum    & Lorem Ipsum         & Lorem Ipsum     & Lorem Ipsum      \\ \hline
%        U\_T\_02    & Lorem Ipsum          & Lorem Ipsum    & Lorem Ipsum         & Lorem Ipsum     & Lorem Ipsum      \\ \hline
%        U\_T\_03    & Lorem Ipsum          & Lorem Ipsum    & Lorem Ipsum         & Lorem Ipsum     & Lorem Ipsum      \\ \hline
%        U\_T\_04    & Lorem Ipsum          & Lorem Ipsum    & Lorem Ipsum         & Lorem Ipsum     & Lorem Ipsum      \\ \hline
%        U\_T\_05    & Lorem Ipsum          & Lorem Ipsum    & Lorem Ipsum         & Lorem Ipsum     & Lorem Ipsum      \\ \hline
%    \end{tabularx}
%\end{table}


\begin{table}[ht!]
    \centering
    \caption{Admin Test case}
    \begin{tabularx}{\textwidth}{|c|>{\RaggedRight}p{2.2cm}|>{\RaggedRight}X|>{\RaggedRight}p{2.7cm}|>{\RaggedRight}p{1.98cm}|>{\RaggedRight}p{1.5cm}|}
        \hline
        \textbf{Id} & \textbf{Description} & \textbf{Input} & \textbf{Expected Output} & \textbf{Output} & \textbf{Outcome} \\
        \hline
        \multirow{2}{*}{A\_T\_01} & \multirow{2}{2.2cm}{Admin Login} & \multirow{2}{\linewidth}{Email: admin@example.com \\ Password: Adm1nP@ss (valid credentials)} & Login Successfully  & Redirected to admin dashboard & Passed \\
        & & & & & \\
        \hline
        \multirow{2}{*}{A\_T\_02} & \multirow{2}{2.2cm}{Admin Login 2} & Email: invalid@example.com & \multirow{2}{\linewidth}{Error message displayed} & Unable to login & Passed \\
        & & Password: invalid (invalid credentials) & & & \\
        \hline
        A\_T\_03 & Creating New Configuration & Adding all necessary values to the new FL configuration form & Configuration Created successfully & Configuration Created successfully & Passed \\
        \hline
        A\_T\_04 & Creating New Configuration 2 & Entering all values except stop condition & Error of missing value message displayed & Configuration is not created & Passed \\
        \hline
        A\_T\_05 & Creating New Experiment & Name is written and FL configuration is selected & Experiment Created successfully & Experiment Created successfully & Passed \\
        \hline
        A\_T\_06 & Starting an Experiment & Press Start Experiment button & Experiment starts & Experiment starts & Passed \\
        \hline
        \multirow{2}{*}{A\_T\_07} & \multirow{2}{2.2cm}{Search Configuration by name} & Write name with existing configuration name & \multirow{2}{\linewidth}{Show the list of the configuration with that name} & List of configurations with that name  & Passed \\
        \hline
        \multirow{2}{*}{A\_T\_08} & \multirow{2}{2.2cm}{Search experiment by name} & Write name with existing experiment name & \multirow{2}{\linewidth}{Show the list of the experiments with that name} & List of experiments with that name  & Passed \\
        \hline
    \end{tabularx}
\end{table}


\begin{table}[ht!]
    \centering
    \caption{User Test case}
    \begin{tabularx}{\textwidth}{|c|>{\RaggedRight}p{2.2cm}|>{\RaggedRight}X|>{\RaggedRight}p{2.7cm}|>{\RaggedRight}p{1.98cm}|>{\RaggedRight}p{1.5cm}|}
        \hline
        \textbf{Id} & \textbf{Description} & \textbf{Input} & \textbf{Expected Output} & \textbf{Output} & \textbf{Outcome} \\
        \hline
        \multirow{2}{*}{U\_T\_01} & \multirow{2}{2.2cm}{User Login} & \multirow{2}{\linewidth}{Email: firstTest@example.com \\ Password: P@ssw0rd (valid credentials)} & Login Successfully  & Redirected to user dashboard & Passed \\
        & & & & & \\
        \hline
        \multirow{2}{*}{U\_T\_02} & \multirow{2}{2.2cm}{User Login 2} & Email: wrong@example.com & \multirow{2}{\linewidth}{Error message displayed} & Unable to login & Passed \\
        & & Password: invalid (invalid credentials) & & & \\
        \hline
        \multirow{2}{*}{U\_T\_03} & \multirow{2}{2.2cm}{User Signup} & Email: new@example.com & Sign up successfully & Redirected to user dashboard & Passed \\
        & & Password: P@ssw0rd (valid input) & & & \\
        \hline
        \multirow{2}{*}{U\_T\_04} & \multirow{2}{2.2cm}{User Signup 2} & Email: new@example.com & \multirow{2}{\linewidth}{Error message displayed} & Unable to Signup & Passed \\
        & & Password: invalid (invalid password) & & & \\
        \hline
        \multirow{2}{*}{U\_T\_05} & \multirow{2}{2.2cm}{Search Configuration by name} & Write name with existing configuration name & \multirow{2}{\linewidth}{Show the list of the configuration with that name} & List of configurations with that name  & Passed \\
        \hline
        \multirow{2}{*}{U\_T\_06} & \multirow{2}{2.2cm}{Search experiment by name} & Write name with existing experiment name & \multirow{2}{\linewidth}{Show the list of the experiments with that name} & List of experiments with that name  & Passed \\
        \hline
    \end{tabularx}
\end{table}




