\chapter{Conclusion}
%In this chapter, we summarize the key points of the document and discuss possible future directions for the project.


The project has delivered a robust web console for initiating federated learning experiments 
concurrently across remote environments. Leveraging Java Spring Boot with MVC pattern, 
WebSocket technology, and MongoDB as the backend database, a dynamic platform for real-time 
analytics and seamless communication between frontend and backend systems has been created.
\\
The implementation choices have proven effective, facilitating efficient communication and 
coordination between disparate components. By utilizing JInterface as a middleware to interface 
with the infrastructure managing federated learning experiments, seamless integration and 
enhanced interoperability have been achieved.
\\
Moreover, the design approach enables scalability and extensibility, laying the groundwork 
for future enhancements and expansions. The flexibility of the web console architecture allows 
for the incorporation of new experiment algorithms and the refinement of message structures to 
further optimize performance and functionality.
\\
Overall, the implementation demonstrates a well-thought-out approach that effectively meets 
the project's objectives. The successful deployment of the web console signifies not only the 
strength of the implementation choices but also the potential for continued innovation and 
growth in the realm of federated learning experimentation.
